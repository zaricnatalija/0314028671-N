\documentclass[30pt, letterpaper,utf8,a4paper]{article}
\usepackage{graphicx}
\graphicspath{{slike/}}
\usepackage{float}
\usepackage[croatian]{babel}
\title{Konfiguracija Računala za Web Dizajniranje}
\author{Natalija Žarić}
\date{\today}

\begin{document}
\maketitle
\newpage


\tableofcontents



\newpage


\section{Uvod}
U ovom radu ćemo opisati  konfiguraciju osobnog računala namijenjenog za web dizajniranje. Modernih i vizaolno privlačnih web stranica
\section{Komponente Računala}






\subsection{Procesor}
\begin{itemize}
    \item Model: AMD Ryzen 9 5900X
    \item Broj jezgri: 12
    \item Radni takt: 3.7 GHz , 4.8 GHz 
    \item Cijena: 500 EUR
    \item Fotografija: \begin{figure}[H]
        \centering
        \includegraphics[width=0.4\textwidth]{amd_ryzen.jpg}
        \caption{AMD Ryzen 9 5900X}
    \end{figure}
    \item \textbf{Obrazloženje:} Odabran je zbog visokih performansi potrebnih za obradu grafičkih sadržaja u web dizajnu.
\end{itemize}

\subsection{Grafička Kartica}
\begin{itemize}
    \item Model: NVIDIA GeForce RTX 3070
    \item VRAM: 8GB GDDR6
    \item Cijena: 700 EUR
    \item Fotografija: \begin{figure}[H]
        \centering
        \includegraphics[width=0.4\textwidth]{nvidia_rtx.jpg}
        \caption{NVIDIA GeForce RTX 3070}
    \end{figure}
    \item \textbf{Obrazloženje:} Grafika s visokim kapacitetom VRAM-a za obradu složenih vizualnih elemenata.
\end{itemize}





\subsection{Pohrana (HDD/SSD)}
\begin{itemize}
    \item SSD: Samsung 970 EVO Plus, 1TB
    \item HDD: Seagate Barracuda, 2TB, 7200RPM, 64MB cache, SATA3
      \item Cijena: 72 ,58 €

    \item\textbf{Obrazloženje:} SSD za brz operativni sustav i aplikacije, HDD za pohranu velikih datoteka i projekata.
\end{itemize}


\subsection{RAM Memorija}
\begin{itemize}
\item Model: Corsair Vengeance LPX DDR4 32GB (2 x 16GB)
    \item Kapacitet: 32GB DDR4 (2 x 16GB)
    \item Brzina: 3200 MHz
    \item Cijena: 150 EUR
    \item \textbf{Obrazloženje:} Brza i pouzdana RAM memorija za rad s velikim datotekama i aplikacijama.
\end{itemize}



\subsection{Matična ploča - ASUS ROG Strix X570-E Gaming}
\begin{itemize}
    \item {Čipset:} AMD X570
    \item Soket: AM4
      \item Cijena: 340
    \item {Memorija:} Do 128GB DDR4, 4800 MHz 
    \item {USB priključci:} USB 3.2 Gen 2
\begin{figure}[h]
    \centering
    \includegraphics[width=0.7\textwidth]{maticna.jpg}
    \caption{ASUS ROG Strix X570-E Gaming}
        \end{figure}
\item \textbf{Obrazloženje:} Kvalitetna X570 ploča koja podržava Ryzen 9 5900X, PCIe 4.0, Wi-Fi 6, i ima napredne značajke hlađenja.

\end{itemize}



\subsection{Napajanje}
\begin{itemize}
  \item \textbf{Model:} Corsair RM750x
  \item \textbf{Snaga:} 750 W
  \item Cijena: 183€
  \item \textbf{Slika:}
    \begin{figure}[h]
      \centering
      \includegraphics[width=0.4\textwidth]{ napajanje.jpg}
      \caption{Corsair RM750x Napajanje}
    \end{figure}
  \item \textbf{Obrazloženje:} Corsair RM750x sam uzela  zbog pouzdanosti i dovoljne snage za podršku visokim performansama ostalih komponenti.
\end{itemize}
\newpage



\subsection{Hlađenje}
\begin{itemize}
\item \textbf{Vodeno Hlađenje:} NZXT Kraken X62
\item Cijena: 1.171 €
\begin{figure}[h]
 \centering
 \includegraphics[width=0.4\textwidth]{hladjenje2.jpg}
\caption{ NZXT Kraken X62 }
\end{figure}
 \item \textbf{Obrazloženje:} Kvalitetno i dobro hlađenje 
\end{itemize}
\newpage

\section{Periferija}
\subsection{Monitor}
\begin{itemize}
    \item Model: Dell UltraSharp UP2716D
 \item Veličina ekrana: 27 inča
\item Rezolucija: 2560x1440
\item Cijena: 600 EUR
    \item Fotografija: \begin{figure}[H]
        \centering
        \includegraphics [width=0.5\textwidth]{dell_monitor.jpg}
        \caption{Dell UltraSharp Monitor}
          \item \textbf{Obrazloženje:} Monitor s visokom razlučivošću (2560x1440), preciznim bojama i dobrom reprodukcijom sRGB i Adobe RGB spektra boja.

    \end{figure}
\end{itemize}




\subsection{Tipkovnica i Miš}
\begin{itemize}
\item Tipkovnica: Logitech MX Keys
\item Miš: Logitech MX Master 3
 \item Cijena (zajedno): 150 EUR
 \item Fotografije: 
 \begin{figure}[H]
\centering  
\includegraphics[width=6cm]{Logitech MX Keys.jpg}
  \hfill
\includegraphics[width=5cm]{logitech_mxmaster.jpg}
\caption{Logitech MX Master 3}
\item \textbf{Obrazloženje:}Kvalitetan bežični miš i tipkovnica za udoban rad.
\end{figure}
\end{itemize}



\subsection{Kućište}
\begin{itemize}
    \item Model: NZXT H510
    \item Tip: Mid Tower
    \item Cijena: 80 EUR
    \item Fotografija: \begin{figure}[H]
        \centering
        \includegraphics[width=0.4\textwidth]{kuciste.jpg}
        \caption{NZXT H510 Mid Tower}
    \end{figure}
    \item Obrazloženje: Moderan dizajn, dobra ventilacija i dovoljno prostora za vaše komponente.
\end{itemize}



\newpage

\section{Ukupna Cijena}
Ukupna cijena računala iznosi 2834 EUR.

\section{Izvor Informacija}
Sve cijene su preuzete s web stranice za kupnju komponenata za računala

\end{document}

